\documentclass[12pt]{article}

\usepackage[owncaptions]{vhistory}
\usepackage[table]{xcolor}
\usepackage{array}
\usepackage{multirow}
\usepackage{tabularx}
\usepackage{makecell}
\usepackage{environ}
\usepackage{shareboard}

\def\progetto{Shareboard}
\def\riferimento{}
\def\presentatoda{Antonio Romano (AR), Alessandro Saverio De Maio (AM), Carmine Leo (CL)}
\def\approvatoda{}
\def\destinatario{Studenti di Ingegneria del Software 2021/22}
\def\footer{RAD}

\title{Requirements\\Elicitation}
\author{\presentatoda}

\newcounter{scenario}
\newcounter{scenariostep}[scenario]
\counterwithin{scenariostep}{scenario}
\newcommand{\refcounter}[1]{\addtocounter{#1}{-1}\refstepcounter{#1}}

\NewEnviron{scenario}[1]{
    \stepcounter{scenario}
    \newcommand{\system}[1]{
      \stepcounter{scenariostep}
      \thescenariostep & System: & \refcounter{scenariostep}##1\\
      \hline
    }
    \newcommand{\actor}[1]{
      \stepcounter{scenariostep}\thescenariostep & Actor: & \refcounter{scenariostep}##1\\
      \hline
    }
    \begin{tabularx}{\linewidth}{|c|c|X|}
        \hline
        \multicolumn{3}{|c|}{\textbf{#1}}\\
        \hline
        \BODY
        \multicolumn{3}{|c|}{}\\
        \hline
    \end{tabularx}\par\vskip-1.4pt
}

\NewEnviron{note}{
    \newcommand{\entry}[2]{\multicolumn{2}{|c|}{##1} &  ##2\\\hline}
    \begin{tabularx}{\linewidth}{|c|c|X|}
        \multicolumn{3}{|c|}{\textbf{Note}}\\
        \hline
        \BODY
        \multicolumn{3}{|c|}{}\\
        \hline
    \end{tabularx}\par\vskip-1.4pt
}

\NewEnviron{specialreq}{
    \newcounter{entry}
    \newcommand{\entry}[1]{\multicolumn{2}{|c|}{\stepcounter{entry}\theentry} & ##1\\\hline}
    \begin{tabularx}{\linewidth}{|c|c|X|}
        \hline
        \multicolumn{3}{|c|}{\textbf{Special Requirements}}\\
        \hline
        \BODY
        \multicolumn{3}{|c|}{}\\
        \hline
    \end{tabularx}
}

\NewEnviron{usecase}{
    \sffamily
    \centering
    \setcounter{scenario}{0}
    \setcounter{scenariostep}{0}
    \newcommand{\makeheader}{
        \begin{tabularx}{\linewidth}{|c|X|c|c|}
        \hline
        \textbf{Identificativo} & \nome & \textbf{Data}  & \data\\
        \cline{3-4}
        \id&&\textbf{Vers.}&\versione\\
        \cline{3-4}
        &&\textbf{Autore}&\autore\\
        \hline
        \textbf{Descrizione} & \multicolumn{3}{c|}{\descrizione}\\
        \hline
        \textbf{Attore principale} & \multicolumn{3}{c|}{\makecell{\attoreprincipale}}\\
        \hline
        \textbf{Attori secondari} & \multicolumn{3}{c|}{\makecell{\attorisecondari}}\\
        \hline
        \textbf{Entry condition} & \multicolumn{3}{c|}{\entrycondition}\\
        \hline
        \makecell{\textbf{Exit condition}\\ \raggedleft{On success}} & \multicolumn{3}{c|}{\onsuccess}\\
        \hline
        \makecell{\textbf{Exit condition}\\ \raggedleft{On error}} & \multicolumn{3}{c|}{\onerror}\\
        \hline
        \textbf{Rilevanza/User priority} & \multicolumn{3}{c|}{\rilevanza}\\
        \hline
        \textbf{Frequenza stimata} & \multicolumn{3}{c|}{\frequenzastimata}\\
        \hline
        \textbf{Extension point} & \multicolumn{3}{c|}{\extensionpoint}\\
        \hline
        \textbf{Generalization of} & \multicolumn{3}{c|}{\generalizationof}\\
        \hline
        \end{tabularx}\par\vskip-1.4pt
    }
    \BODY
}

\begin{document}

\maketitle

\tableofcontents
\clearpage
\begin{versionhistory}
  \vhEntry{1.0}{21/10/2021}{AR, AM, CL}{Documento creato}
\end{versionhistory}
\clearpage

\section{Introduzione}
\subsection{Scopo del sistema}
\subsection{Ambito del sistema}
\subsection{Obiettivi e criteri di successo del progetto}
\subsection{Definizione, acronimi e abbreviazioni}
\subsection{Riferimenti}
\subsection{Panoramica}
\section{Sistema corrente}
\section{Sistema proposto}
\subsection{Panoramica}
\subsection{Requisiti funzionali}
\subsection{Requisiti non funzionali}
\subsection{Modelli di sistema}
\subsubsection{Scenari}
\subsubsection{Modello dei casi d'uso}

\begin{usecase}
  \def\nome{nome}
  \def\data{data}
  \def\id{id}
  \def\versione{versione}
  \def\autore{autore}
  \def\descrizione{descrizione}
  \def\attoreprincipale{attoreprincipale}
  \def\attorisecondari{attorisecondari}
  \def\entrycondition{entrycondition}
  \def\onsuccess{onsuccess}
  \def\onerror{onerror}
  \def\rilevanza{rilevanza}
  \def\frequenzastimata{frequenzastimata}
  \def\extensionpoint{extensionpoint}
  \def\generalizationof{generalizationof}

  \makeheader

  \begin{scenario}{Main scenario}
    \system{Sono un'azione di sistema}
    \actor{Sono un'azione da parte dell'attore}
    \actor{
      \label{azione:mia}
      lorem ipsum bla bla
      \begin{enumerate}
        \item lista da parte dell'attore
        \item wow
      \end{enumerate}
    }
  \end{scenario}
  \begin{scenario}{Scenario alternativo}
    \system{ciao}
    \actor{questo è uno step}
  \end{scenario}
  \begin{note}
    \entry{NUMERO}{bla bla bla}
    \entry{\ref{azione:mia}}{<- dimostrazione di label e ref.}
  \end{note}
  \begin{specialreq}
    \entry{test test test}
  \end{specialreq}
\end{usecase}\\

\begin{tabularx}{\linewidth}{|c|X|}
  \hline
  Nome scenario: & SC\_1\_Login\\ 
  \hline
  Partecipanti:  & Giovanni:utente\\
  \hline
  Flusso degli eventi: & \raggedright Giovanni desidera interagire con la community quindi necessita di effettuare il login.
  {\begin{tabularx}{\linewidth}{|c|X|}
    \hline
    Actor Steps & System Steps\\
    \hline
    test & test\\
    \hline
  \end{tabularx}}\\
\end{tabularx}


\subsubsection{Modello ad oggetti}
\subsubsection{Modello dinamico}
\subsubsection{Interfaccia utente - percorso navigazionale e mock-up grafico}
\section{Glossario}

\end{document}