\documentclass[12pt]{article}

\usepackage[owncaptions]{vhistory}
\usepackage[table]{xcolor}
\usepackage{array}
\usepackage{multirow}
\usepackage{tabularx}
\usepackage{makecell}
\usepackage{environ}
\usepackage{shareboard}

\def\progetto{Shareboard}
\def\riferimento{}
\def\presentatoda{Antonio Romano (AR), Alessandro Saverio De Maio (AM), Carmine Leo (CL)}
\def\approvatoda{}
\def\destinatario{Studenti di Ingegneria del Software 2021/22}
\def\footer{RAD}

\title{Requirements\\Elicitation}
\author{\presentatoda}

\newcounter{scenario}
\newcounter{scenariostep}[scenario]
\counterwithin{scenariostep}{scenario}
\newcommand{\refcounter}[1]{\addtocounter{#1}{-1}\refstepcounter{#1}}

\NewEnviron{scenario}[1]{
    \stepcounter{scenario}
    \newcommand{\system}[1]{
      \stepcounter{scenariostep}
      \thescenariostep & System: & \refcounter{scenariostep}##1\\
      \hline
    }
    \newcommand{\actor}[1]{
      \stepcounter{scenariostep}\thescenariostep & Actor: & \refcounter{scenariostep}##1\\
      \hline
    }
    \begin{tabularx}{\linewidth}{|c|c|X|}
        \hline
        \multicolumn{3}{|c|}{\textbf{#1}}\\
        \hline
        \BODY
        \multicolumn{3}{|c|}{}\\
        \hline
    \end{tabularx}\par\vskip-1.4pt
}

\NewEnviron{note}{
    \newcommand{\entry}[2]{\multicolumn{2}{|c|}{##1} &  ##2\\\hline}
    \begin{tabularx}{\linewidth}{|c|c|X|}
        \multicolumn{3}{|c|}{\textbf{Note}}\\
        \hline
        \BODY
        \multicolumn{3}{|c|}{}\\
        \hline
    \end{tabularx}\par\vskip-1.4pt
}

\NewEnviron{specialreq}{
    \newcounter{entry}
    \newcommand{\entry}[1]{\multicolumn{2}{|c|}{\stepcounter{entry}\theentry} & ##1\\\hline}
    \begin{tabularx}{\linewidth}{|c|c|X|}
        \hline
        \multicolumn{3}{|c|}{\textbf{Special Requirements}}\\
        \hline
        \BODY
        \multicolumn{3}{|c|}{}\\
        \hline
    \end{tabularx}
}

\NewEnviron{usecase}{
    \sffamily
    \centering
    \setcounter{scenario}{0}
    \setcounter{scenariostep}{0}
    \newcommand{\makeheader}{
        \begin{tabularx}{\linewidth}{|c|X|c|c|}
        \hline
        \textbf{Identificativo} & \nome & \textbf{Data}  & \data\\
        \cline{3-4}
        \id&&\textbf{Vers.}&\versione\\
        \cline{3-4}
        &&\textbf{Autore}&\autore\\
        \hline
        \textbf{Descrizione} & \multicolumn{3}{c|}{\descrizione}\\
        \hline
        \textbf{Attore principale} & \multicolumn{3}{c|}{\makecell{\attoreprincipale}}\\
        \hline
        \textbf{Attori secondari} & \multicolumn{3}{c|}{\makecell{\attorisecondari}}\\
        \hline
        \textbf{Entry condition} & \multicolumn{3}{c|}{\entrycondition}\\
        \hline
        \makecell{\textbf{Exit condition}\\ \raggedleft{On success}} & \multicolumn{3}{c|}{\onsuccess}\\
        \hline
        \makecell{\textbf{Exit condition}\\ \raggedleft{On error}} & \multicolumn{3}{c|}{\onerror}\\
        \hline
        \textbf{Rilevanza/User priority} & \multicolumn{3}{c|}{\rilevanza}\\
        \hline
        \textbf{Frequenza stimata} & \multicolumn{3}{c|}{\frequenzastimata}\\
        \hline
        \textbf{Extension point} & \multicolumn{3}{c|}{\extensionpoint}\\
        \hline
        \textbf{Generalization of} & \multicolumn{3}{c|}{\generalizationof}\\
        \hline
        \end{tabularx}\par\vskip-1.4pt
    }
    \BODY
}

\begin{document}

\maketitle

\tableofcontents
\clearpage
\begin{versionhistory}
  \vhEntry{1.0}{21/10/2021}{AR, AM, CL}{Documento creato}
\end{versionhistory}
\clearpage

\section{Introduzione}
\subsection{Scopo del sistema}

\subsection{Ambito del sistema}
\subsection{Obiettivi e criteri di successo del progetto}
\subsection{Definizione, acronimi e abbreviazioni}
\subsection{Riferimenti}
\subsection{Panoramica}
\section{Sistema corrente}
\section{Sistema proposto}
\subsection{Panoramica}
\subsection{Requisiti funzionali}
\subsection{Requisiti non funzionali}
\subsection{Modelli di sistema}
\subsubsection{Scenari}

\begin{tabularx}{\linewidth}{|c|X|}
  \hline
  Nome scenario: & SC\_1\_Login\\ 
  \hline
  Partecipanti:  & Giovanni:utente\\
  \hline
  Flusso degli eventi: & \raggedright Giovanni desidera interagire con la community quindi necessita di effettuare il login.
  {\begin{tabularx}{\linewidth}{|X|X|}
    \hline
    Actor Steps & System Steps\\
    \hline
    1. Giovanni inserisce le sue credenziali d’accesso & 2.	Il sistema verifica nel repository se le credenziali sono corrette e autentica Giovanni.\\
    \hline
  \end{tabularx}}\\
\end{tabularx}

\bigskip

\begin{tabularx}{\linewidth}{|c|X|}
  \hline
  Nome scenario: & SC\_2\_Registrazione\\ 
  \hline
  Partecipanti:  & Rita:utente\\
  \hline
  Flusso degli eventi: & \raggedright Rita scopre la community di Shareboard e desidera farne parte, decide di registrarsi in modo da poter interagire con la community.
  {\begin{tabularx}{\linewidth}{|X|X|}  
    \hline
    Actor Steps & System Steps\\
    \hline
    1. Rita inserisce le informazioni richieste nell'apposito form dedicato alla registrazione. & 2. Il sistema verifica che le informazioni inserite siano corrette e le memorizza.\\
    \hline
  \end{tabularx}}\\
\end{tabularx}

\bigskip

\begin{tabularx}{\linewidth}{|c|X|}
  \hline
  Nome scenario: & SC\_3\_Ban\\ 
  \hline
  Partecipanti:  & Lucia:admin, Marco:utente\\
  \hline
  Flusso degli eventi: & \raggedright Lucia intende bandire Marco, un utente che non ha rispettato le linee guida della community.
  {\begin{tabularx}{\linewidth}{|X|X|}
    \hline
    Actor Steps & System Steps\\
    \hline
    1.	Lucia inserisce le sue credenziali d’accesso. & 2.	Il sistema verifica nel repository se le credenziali sono corrette e autentica Lucia come admin.\\
    \hline
    3.	Lucia apre il pannello admin e si dirige nella gestione utenti. & 4.	Il sistema mostra la lista di utenti.\\
    \hline
    5.	Lucia identifica l’utente e seleziona l’apposita azione per bandire un utente. & 6.	Il sistema risponde con la pagina che permette di specificare la durata del ban.\\
    \hline
    7.	Lucia inserisce la data di inizio e fine del ban. & 8.	Il sistema memorizza la durata del ban.\\
    \hline
  \end{tabularx}}\\
\end{tabularx}

\bigskip

\begin{tabularx}{\linewidth}{|c|X|}
  \hline
  Nome scenario: & SC\_4\_Voto\\ 
  \hline
  Partecipanti:  & Tony:utente\\
  \hline
  Flusso degli eventi: & \raggedright Tony visualizzando un post decide di assegnare un voto positivo perché considera interessanti i contenuti riportati.
  {\begin{tabularx}{\linewidth}{|X|X|}  
    \hline
    Actor Steps & System Steps\\
    \hline
    1. Tony effettua il login inserendo le proprie credenziali nel form di login. & 2. Il sistema verifica nel repository se le credenziali sono corrette e autentica Tony.\\
    \hline
    3. Tony seleziona un post dalla sua homepage. & 4. Il sistema riceve la richiesta e mostra il post selezionato in una pagina dedicata.\\
    \hline
    5. Tony ritiene interessanti i contenuti del post e decide di assegnargli un voto positivo. & 6. Il sistema riceve la richiesta e incrementa il contatore di voti positivi.\\
    \hline
  \end{tabularx}}\\
\end{tabularx}

\bigskip

\begin{tabularx}{\linewidth}{|c|X|}
  \hline
  Nome scenario: & SC\_5\_CommentoPost\\ 
  \hline
  Partecipanti:  & Sara:utente\\
  \hline
  Flusso degli eventi: & \raggedright Sara visualizzando un post decide di commentare per aggiungere un suo parere sui contenuti.
  {\begin{tabularx}{\linewidth}{|X|X|}  
    \hline
    Actor Steps & System Steps\\
    \hline
    1. Sara effettua il login inserendo le proprie credenziali nel form di login. & 2. Il sistema verifica nel repository se le credenziali sono corrette e autentica Sara.\\
    \hline
    3. Sara seleziona un post dalla sua homepage. & 4. Il sistema riceve la richiesta e mostra il post selezionato in una pagina dedicata.\\
    \hline
    5. Sara desidera aggiungere delle sue opinioni sull'argomento quindi scrive il testo da condividere e lo invia al sistema. & 6. Il sistema riceve il testo inserito da Sara verificando che il numero di caratteri sia corretto e lo memorizza per pubblicarlo nel post.\\
    \hline
  \end{tabularx}}\\
\end{tabularx}

\bigskip

\begin{tabularx}{\linewidth}{|c|X|}
  \hline
  Nome scenario: & SC\_6\_CreazioneSezione\\ 
  \hline
  Partecipanti:  & Joel:admin\\
  \hline
  Flusso degli eventi: & \raggedright Joel, admin della community Shareboard, nota la mancanza di una sezione che potrebbe interessare un gran numero di utenti e decide quindi di crearla.
  {\begin{tabularx}{\linewidth}{|X|X|}  
    \hline
    Actor Steps & System Steps\\
    \hline
    1. Joel effettua il login inserendo le proprie credenziali nel form di login. & 2. Il sistema verifica nel repository se le credenziali sono corrette e autentica Jeol come admin.\\
    \hline
    3. Joel apre il pannello dedicato agli admin. & 4. Il sistema riceve la richiesta e mostra il pannello admin.\\
    \hline
    5. Joel seleziona l'apposita funzione per la creazione di una sezione. & 6. Il sistema mostra il form che permette di inserire le informazioni necessarie alla creazione della sezione.\\
    \hline
    7. Joel inserisce le informazioni richieste e le invia al sistema. & 8. Il sistema mostra a Joel la pagina di gestione delle sezioni contenente la sezione da lui creata.\\
    \hline
  \end{tabularx}}\\
\end{tabularx}

\bigskip

\begin{tabularx}{\linewidth}{|c|X|}
  \hline
  Nome scenario: & SC\_7\_CancellazioneSezione\\ 
  \hline
  Partecipanti:  & Antonella:admin\\
  \hline
  Flusso degli eventi: & \raggedright Antonella, admin della community Shareboard, nota che una particolare sezione non genera interesse nella community e decide quindi di eliminarla.
  {\begin{tabularx}{\linewidth}{|X|X|}  
    \hline
    Actor Steps & System Steps\\
    \hline
    1. Antonella effettua il login inserendo le proprie credenziali nel form di login. & 2. Il sistema verifica nel repository se le credenziali sono corrette e autentica Antonella come admin.\\
    \hline
    3. Antonella apre il pannello dedicato agli admin. & 4. Il sistema riceve la richiesta e mostra il pannello admin.\\
    \hline
    5. Antonella apre la pagina che permette di gestire le sezioni. & 6. Il sistema mostra la lista di sezioni attualmente presenti.\\
    \hline
    7. Antonella seleziona l'apposita funzione per eliminare la sezione. & 8. Il sistema chiede conferma ad Antonella ed elimina la sezione dal repository.\\
    \hline
  \end{tabularx}}\\
\end{tabularx}

\subsubsection{Modello dei casi d'uso}

\begin{usecase}
  \def\nome{Login}
  \def\data{26/10/2021}
  \def\id{UC\_1}
  \def\versione{0.1}
  \def\autore{Leo Carmine}
  \def\descrizione{Lo UC fornisce la funzionalità per il login di un utente.}
  \def\attoreprincipale{Utente}
  \def\attorisecondari{NA}
  \def\entrycondition{È visualizzato il comando per il login.}
  \def\onsuccess{L'utente risulta registrato ed effettua il login con successo.}
  \def\onerror{Accesso negato all'utente}
  \def\rilevanza{Hight}
  \def\frequenzastimata{1000 usi/giorno}
  \def\extensionpoint{NA}
  \def\generalizationof{NA}

  \makeheader

  \begin{scenario}{Flusso di eventi principale/Main scenario}
    \actor{Richiede di effettuare il login tramite il comando apposito.}
    \system{
      \label{azione:FormV}
    Visualizza un form che richiede l'inserimento di:
      \begin{itemize}
        \item Username: Stringa di massimo 30 caratteri.
        \item Password: Stringa compresa tra 3 e 30 caratteri.
      \end{itemize}
    }
    \actor{Riempie tutti i campi richiesti e sottomette la form compilata}
    \system{
      \label{azione:Check}
      Verifica che:
      \begin{itemize}
        \item tutti i campi obbligatori siano stati compilati.
        \item la stringa username sia di massimo 30 caratteri.
        \item la password sia compresa tra 3 e 30 caratteri.
        \item lo username sia presente nel sistema.
        \item la password inserita corrisponda allo username.
      \end{itemize}
    }
    \system{Mostra la homepage con l'accesso effettuato.}
  \end{scenario}
  \begin{scenario}{Flusso di eventi alternativo: Un campo obbligatorio non è stato compilato}
    \system{Visualizza un messaggio di errore che segnala all'utente che non ha inserito tutti i campi obbligatori.}
    \system{Resta in attesa di una nuova sottomissione del form.}
  \end{scenario}
  \begin{scenario}{Flusso di eventi alternativo: La password è inferiore ai 3 caratteri}
    \system{Visualizza un messaggio di errore che segnala all'utente che la password inserita è inferiore ai 3 caratteri.}
    \system{Resta in attesa di una nuova sottomissione del form.}
  \end{scenario}
  \begin{scenario}{Flusso di eventi alternativo: Le credenziali inserite non sono valide}
    \system{Visualizza un messaggio di errore che segnala all'utente che le credenziali inserite non sono valide.}
    \system{Resta in attesa di una nuova sottomissione del form.}
  \end{scenario}
  \begin{scenario}{Flusso di eventi di ERRORE: Da definire}
    \system{NA}
    \system{NA}
  \end{scenario}
  \begin{note}
    \entry{NUMERO}{bla bla bla}
    \entry{\ref{azione:FormV}}{<- dimostrazione di label e ref.}
  \end{note}
  \begin{specialreq}
    \entry{test test test}
  \end{specialreq}
\end{usecase}\\


\subsubsection{Modello ad oggetti}
\subsubsection{Modello dinamico}
\subsubsection{Interfaccia utente - percorso navigazionale e mock-up grafico}
\section{Glossario}

\end{document}