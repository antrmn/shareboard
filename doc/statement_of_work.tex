\documentclass[12pt]{article}

\usepackage[owncaptions]{vhistory}
\usepackage[table]{xcolor}
\usepackage{shareboard}

\def\progetto{Shareboard}
\def\riferimento{}
\def\presentatoda{Antonio Romano (AR), Alessandro Saverio De Maio (AM), Carmine Leo (CL)}
\def\approvatoda{}
\def\destinatario{Studenti di Ingegneria del Software 2021/22}
\def\footer{SOW}

\title{Statement of Work\\Progetto}
\author{\presentatoda}

\begin{document}
\maketitle

\begin{versionhistory}
  \vhEntry{1.0}{18/10/2021}{AR, AM, CL}{Documento creato}
\end{versionhistory}
\clearpage

\section{Piano Strategico/Strategic Plan}
  Il team di Shareboard intende promuovere lo scambio di idee e l’interazione tra persone, incoraggiando la curiosità verso argomenti, nuovi o già di proprio interesse, e incentivando gli utenti a proporre contenuti stimolanti.
  
  \section{Obiettivi di Business/Business Needs}
  Il team di Shareboard intende allestire uno spazio di comunicazione che sia accessibile e di uso immediato per i suoi utenti, promuovendo la fruizione e la creazione di contenuti.
  
  \section{Ambito del Prodotto/Product Scope}
  
  \begin{itemize}
  
  \item Un interfaccia semplice ed intuitiva,
  \item Un interfaccia responsive per permetterne l’uso su ogni tipo di dispositivo,
  \item Pubblicazione di contenuti da parte degli utenti,
  \item Possibilità per gli utenti di esprimere opinioni sui contenuti pubblicati da altri utenti,
  \item Possibilità di assegnare un feedback positivo o negativo ai contenuti e alle opinioni degli altri utenti,
  \item Promuovere la visibilità dei contenuti con il maggior numero di feedback positivi.
  
  \end{itemize}
  Scenario 1:
  \emph{Un utente effettua il login, visualizza la home page, decide di interagire con un contenuto, assegnando un voto ed un commento.}
\\
  Scenario 2:
  \emph{Un utente non registrato desidera interagire con la community e quindi si registra inserendo un username univoco e informazioni aggiuntive.}
\\
  Scenario 3:
  \emph{Un utente vuole visualizzare nella homepage i post di una sezione di suo interesse, a tal fine clicca l'apposito tasto relativo alla sezione per poterla aggiungere alla sua lista di sezioni seguite.}
\\
  Scenario 4:
  \emph{Un utente registrato vuole condividere un contenuto che ritiene interessante per la community. Lo posta nell'apposita sezione dove sarà votato e commentato da altri utenti.}
\\
  Scenario 5:
  \emph{Un amministratore aggiunge, modifica o rimuove una sezione.}
\\
  Scenario 6:
  \emph{Un amministratore banna un utente che non ha rispettato le linee guida della community, temporaneamente o permanentemente.}

  
  \section{Data di Inizio e di Fine}
  Inizio: Ottobre 2021
  Fine: Gennaio-Febbraio 2022. E’ possibile concordare la data di consegna che potrà essere una delle seguenti:
  
  \begin{itemize}
      \item I: circa metà Gennaio 2022
      \item II: fine Gennaio 2022
      \item III: prima decade di Febbraio 2022
  \end{itemize}
  
  \section{Deliverables}
  \begin{itemize}
      \item RAD, SDD, ODD, Matrice di Tracciabilità, Test Plan, Test Case Specification, Test incident Report, Test Summary Report, Manuale D’Uso, Manuale Installazione e ogni altro documento richiesto per lo sviluppo del sistema.
  \end{itemize}
  
  \section{Vincoli/Constraints}
  \begin{itemize}
  \item Rispetto scadenze
  \item Budget/Effort non superiore a 50*n ore dove n sono i membri del team 
  \item Uso di tre Design Pattern
  \item Uso di UML
  \item Utilizzo di un sistema di versioning, dove tutti i membri del team forniscono il loro contributo
  \item Utilizzo di Trello per divisione compiti
  \item Utilizzo di Discord, Teams e Telegram per comunicazione
  \item Utilizzo di quality tool come Checkstyle
  \item Parte di progetto con approccio Agile (Scrum)
  \end{itemize}
  
  \section{Criteri di Accettazione/Acceptance Criteria}
  \begin{itemize}
      \item Branch coverage dei casi di test: almeno 75\%
      \item Buona manutenibilità
      \item \colorbox{yellow}{Il numero di warning dati in output da Checkstyle inferiore ad una soglia da definire (molto bassa).}
  \end{itemize}
  
  \section{Criteri di premialità}
  
  \begin{itemize}
      \item Utilizzo di sistemi di build, come Maven;
      \item \colorbox{yellow}{Utilizzo del pull-based development tramite l’applicazione di code review;}
      \item Utilizzo di un processo di Continuous Integration, tramite l’utilizzo di Github Actions.
  \end{itemize}

\end{document}
